
\documentclass[UTF-8]{ctexart}
\usepackage{amsmath,amsfonts,amssymb,bm,indentfirst,hyperref}
\usepackage[cache=false]{minted}
\title{计算物理大作业}
\author{王宇逸\ 谷海桥\ 李禹苏}

\begin{document}
\maketitle
\newpage
\tableofcontents
\newpage

\begin{abstract}
    我们研究了由D.W.Brenner等人开发的分子动力学框架BrennerMD,分析了程序结构、使用算法,并使用程序进行了一定的模拟。
    通过这一过程,更好地理解了分子动力学方法及其应用。
\end{abstract}

\section{框架概述}
BrennerMD是由D.W.Brenner等人使用FORTRAN 77开发的分子动力学开源框架。
尽管框架开源,但是由于年代久远,在互联网上难觅踪迹。
现在唯一能找到源代码的网站在SourceForge.Net上(\url{https://sourceforge.net/projects/brennermd/})。
这个网站给出了源代码,并使用C语言重写了一份。

FORTRAN 77是年代久远的数学编程语言。在那个还使用着打孔纸带的年代,有这样一门对数学运算友好的高级语言,一定是开发人员的首选。
然而时过境迁,在现代人的眼光看来,FORTRAN 77对机器的软硬件所做的语法层面的妥协极不合理,而Fortran也在一次次版本迭代中与曾经的标准面目全非。
因此,使用FORTRAN 77的框架鲜有人问津,只有一些“祖传”的代码因为难读、难改、组织混乱,而一直流传至今。

\section{程序流程与结构}
程序代码文件分为两部分。
Execute文件夹提供了一份能够编译通过的代码,但是这些代码文件平铺在同一个文件夹中,比较混乱;
Subroutines等文件夹又将这些代码文件按照功能分类,并给出了基本的README文件作为介绍。
在Execute文件夹内,运行
\begin{minted}{bash}
    $ gfortran main.f -o main.x -std=legacy -g
\end{minted}
生成main.x可执行程序。
“-g”选项加入调试符号,是为了在程序崩溃(这种情况经常发生)的时候给出清楚的调用栈。

main.f文件内是程序的主函数,大致分为初始化与计算两部分,下面就这两部分分别介绍
\subsection{初始化}
初始化的步骤比较简单,按照如下顺序依次执行:
\begin{minted}{fortran}
    include 'open.inc'
\end{minted}
打开所有文件,并使用写死的、无规律的数字为其编号。
\begin{minted}{fortran}
    call setin
\end{minted}
setin函数初始化一些矩阵与常数。常数包括常用的粒子质量、$\pi$、玻尔兹曼常数等等。
\begin{minted}{fortran}
    call read_data
\end{minted}
read\_data函数读入两个文件的内容:编号为13的input.d与编号为11的coord.d。
input.d记录了一些基本的模拟参数,如步数、记录步长间隔、初始温度、使用算法、使用框架等等,以及可选地更改一些粒子的基本属性。
coord.d的输入输出在同一个文件,用于连续运行程序来模拟。包括文件头、粒子个数、时长、每一步的时间间隔、环境大小、初始坐标、速度、加速度、急动度等。
读入工作有可能有参数不合理的地方,这个时候会直接报错并退出,使用类似
\begin{minted}{fortran}
    include 'close.inc'
    stop
\end{minted}
的模式进行清理并退出。
\begin{minted}{fortran}
    call setpp
    call setpc
    call setgle
\end{minted}
setpp设置势函数的参数,主要是LJ势。
setpc设置预测器校正参数。
setgle设置朗之万参数。
\begin{minted}{fortran}
    call setran
\end{minted}
setran初始化随机数。
首先使用读入的随机数中子pseed,调用setrn进行初始化;之后的随机数调用rannum即可。
pseed也使用rannum更新一次,以保证下一次的种子与这次不同

需要指出,rannum函数虽然接受一个参数,但是它并没有使用。
更有趣的一点是,当我们搜索这个算法的时候,只能找到相同的代码,甚至注释都一样。
那些函数无一例外都有这个没有被使用的参数。

\subsection{计算}

\section{程序算法}

\section{程序应用}
给出的代码中,已经提供了金刚石晶体的数据以供模拟。我们通过更换算法与温度可以得到不同的结果。

\section{总结}
我们研究了BrennerMD这一框架,对于FORTRAN 77有了更深的了解,同时也有一次对于分子动力学模拟有了更深刻的认识。
同时,我们也清楚地认识到,作为一名物理学研究者,使用程序语言清楚地表达自己地算法和思路是多么重要。

通过对于BrennerMD框架的阅读与对于FORTRAN 77语言的学习,我们本着对于历史人物的同情得出这样的结论:
FORTRAN 77的设计使得它完全无法胜任如此大规模程序的维护与扩展,同时,BrennerMD程序的组织与代码的混乱也给这一境况雪上加霜。
这个框架之所以用FORTRAN 77编写,完全是因为当时没有相同性能的其它高级编程语言,而不是因为FORTRAN 77自身的设计合理。
虽然在现代科学研究中,我们不再如此需要关注底层的框架是如何编写的,但是BrennerMD的bug过多,又不够健壮(robust),显然是不能胜任“基础框架”这一任务的。
这个框架除了用于教学,已经没有特别的科研价值了。

\section{分工}
\paragraph{王宇逸}概述、初始化、应用
\paragraph{谷海桥}预测修正
\paragraph{李禹苏}能量最小化

\section{致谢}
感谢助教的帮助与同学们在群里的讨论。

感谢Intel。尽管我们并没有用到他们的编译器,但是不同编译器的结果不同,证明了FORTRAN 77写程序是多么不易。

\end{document}
